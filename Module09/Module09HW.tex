% Options for packages loaded elsewhere
\PassOptionsToPackage{unicode}{hyperref}
\PassOptionsToPackage{hyphens}{url}
%
\documentclass[
]{article}
\usepackage{lmodern}
\usepackage{amssymb,amsmath}
\usepackage{ifxetex,ifluatex}
\ifnum 0\ifxetex 1\fi\ifluatex 1\fi=0 % if pdftex
  \usepackage[T1]{fontenc}
  \usepackage[utf8]{inputenc}
  \usepackage{textcomp} % provide euro and other symbols
\else % if luatex or xetex
  \usepackage{unicode-math}
  \defaultfontfeatures{Scale=MatchLowercase}
  \defaultfontfeatures[\rmfamily]{Ligatures=TeX,Scale=1}
\fi
% Use upquote if available, for straight quotes in verbatim environments
\IfFileExists{upquote.sty}{\usepackage{upquote}}{}
\IfFileExists{microtype.sty}{% use microtype if available
  \usepackage[]{microtype}
  \UseMicrotypeSet[protrusion]{basicmath} % disable protrusion for tt fonts
}{}
\makeatletter
\@ifundefined{KOMAClassName}{% if non-KOMA class
  \IfFileExists{parskip.sty}{%
    \usepackage{parskip}
  }{% else
    \setlength{\parindent}{0pt}
    \setlength{\parskip}{6pt plus 2pt minus 1pt}}
}{% if KOMA class
  \KOMAoptions{parskip=half}}
\makeatother
\usepackage{xcolor}
\IfFileExists{xurl.sty}{\usepackage{xurl}}{} % add URL line breaks if available
\IfFileExists{bookmark.sty}{\usepackage{bookmark}}{\usepackage{hyperref}}
\hypersetup{
  pdftitle={Module 09 HW},
  pdfauthor={Diana McSpadden},
  hidelinks,
  pdfcreator={LaTeX via pandoc}}
\urlstyle{same} % disable monospaced font for URLs
\usepackage[margin=1in]{geometry}
\usepackage{color}
\usepackage{fancyvrb}
\newcommand{\VerbBar}{|}
\newcommand{\VERB}{\Verb[commandchars=\\\{\}]}
\DefineVerbatimEnvironment{Highlighting}{Verbatim}{commandchars=\\\{\}}
% Add ',fontsize=\small' for more characters per line
\usepackage{framed}
\definecolor{shadecolor}{RGB}{248,248,248}
\newenvironment{Shaded}{\begin{snugshade}}{\end{snugshade}}
\newcommand{\AlertTok}[1]{\textcolor[rgb]{0.94,0.16,0.16}{#1}}
\newcommand{\AnnotationTok}[1]{\textcolor[rgb]{0.56,0.35,0.01}{\textbf{\textit{#1}}}}
\newcommand{\AttributeTok}[1]{\textcolor[rgb]{0.77,0.63,0.00}{#1}}
\newcommand{\BaseNTok}[1]{\textcolor[rgb]{0.00,0.00,0.81}{#1}}
\newcommand{\BuiltInTok}[1]{#1}
\newcommand{\CharTok}[1]{\textcolor[rgb]{0.31,0.60,0.02}{#1}}
\newcommand{\CommentTok}[1]{\textcolor[rgb]{0.56,0.35,0.01}{\textit{#1}}}
\newcommand{\CommentVarTok}[1]{\textcolor[rgb]{0.56,0.35,0.01}{\textbf{\textit{#1}}}}
\newcommand{\ConstantTok}[1]{\textcolor[rgb]{0.00,0.00,0.00}{#1}}
\newcommand{\ControlFlowTok}[1]{\textcolor[rgb]{0.13,0.29,0.53}{\textbf{#1}}}
\newcommand{\DataTypeTok}[1]{\textcolor[rgb]{0.13,0.29,0.53}{#1}}
\newcommand{\DecValTok}[1]{\textcolor[rgb]{0.00,0.00,0.81}{#1}}
\newcommand{\DocumentationTok}[1]{\textcolor[rgb]{0.56,0.35,0.01}{\textbf{\textit{#1}}}}
\newcommand{\ErrorTok}[1]{\textcolor[rgb]{0.64,0.00,0.00}{\textbf{#1}}}
\newcommand{\ExtensionTok}[1]{#1}
\newcommand{\FloatTok}[1]{\textcolor[rgb]{0.00,0.00,0.81}{#1}}
\newcommand{\FunctionTok}[1]{\textcolor[rgb]{0.00,0.00,0.00}{#1}}
\newcommand{\ImportTok}[1]{#1}
\newcommand{\InformationTok}[1]{\textcolor[rgb]{0.56,0.35,0.01}{\textbf{\textit{#1}}}}
\newcommand{\KeywordTok}[1]{\textcolor[rgb]{0.13,0.29,0.53}{\textbf{#1}}}
\newcommand{\NormalTok}[1]{#1}
\newcommand{\OperatorTok}[1]{\textcolor[rgb]{0.81,0.36,0.00}{\textbf{#1}}}
\newcommand{\OtherTok}[1]{\textcolor[rgb]{0.56,0.35,0.01}{#1}}
\newcommand{\PreprocessorTok}[1]{\textcolor[rgb]{0.56,0.35,0.01}{\textit{#1}}}
\newcommand{\RegionMarkerTok}[1]{#1}
\newcommand{\SpecialCharTok}[1]{\textcolor[rgb]{0.00,0.00,0.00}{#1}}
\newcommand{\SpecialStringTok}[1]{\textcolor[rgb]{0.31,0.60,0.02}{#1}}
\newcommand{\StringTok}[1]{\textcolor[rgb]{0.31,0.60,0.02}{#1}}
\newcommand{\VariableTok}[1]{\textcolor[rgb]{0.00,0.00,0.00}{#1}}
\newcommand{\VerbatimStringTok}[1]{\textcolor[rgb]{0.31,0.60,0.02}{#1}}
\newcommand{\WarningTok}[1]{\textcolor[rgb]{0.56,0.35,0.01}{\textbf{\textit{#1}}}}
\usepackage{graphicx,grffile}
\makeatletter
\def\maxwidth{\ifdim\Gin@nat@width>\linewidth\linewidth\else\Gin@nat@width\fi}
\def\maxheight{\ifdim\Gin@nat@height>\textheight\textheight\else\Gin@nat@height\fi}
\makeatother
% Scale images if necessary, so that they will not overflow the page
% margins by default, and it is still possible to overwrite the defaults
% using explicit options in \includegraphics[width, height, ...]{}
\setkeys{Gin}{width=\maxwidth,height=\maxheight,keepaspectratio}
% Set default figure placement to htbp
\makeatletter
\def\fps@figure{htbp}
\makeatother
\setlength{\emergencystretch}{3em} % prevent overfull lines
\providecommand{\tightlist}{%
  \setlength{\itemsep}{0pt}\setlength{\parskip}{0pt}}
\setcounter{secnumdepth}{-\maxdimen} % remove section numbering

\title{Module 09 HW}
\author{Diana McSpadden}
\date{10/27/2020}

\begin{document}
\maketitle

\hypertarget{stat-6021-homework-set-9}{%
\section{Stat 6021: Homework Set 9}\label{stat-6021-homework-set-9}}

\hypertarget{h.-diana-mcspadden}{%
\subsection{H. Diana McSpadden}\label{h.-diana-mcspadden}}

\hypertarget{uid-hdm52}{%
\subsection{UID: hdm52}\label{uid-hdm52}}

\hypertarget{date-10302020}{%
\subsection{Date: 10/30/2020}\label{date-10302020}}

\textbf{Attended group with}: Bernhardt, Cheu, Nam, Hiatt, Newman

\hypertarget{question-1.-in-an-experiment-testing-the-effect-of-a-toxic-substance}{%
\section{Question 1. In an experiment testing the effect of a toxic
substance,
\ldots{}}\label{question-1.-in-an-experiment-testing-the-effect-of-a-toxic-substance}}

1,500 insects were divided at random into six groups of 250 each.

The insects in each group were exposed to a fixed dose of the toxic
substance. A day later, the insects were observed. Death from exposure
was recorded as 1, and survival 0.

The file ``insects.txt'' contains the data from the experiment.

\begin{itemize}
\tightlist
\item
  The first column denotes the dose level on a logscale, x;
\item
  the 2nd column denotes the sample size for each group, ni
\item
  ; and the 3rd column denotes the number of insects that died in the
  group, yi
\end{itemize}

You will probably want to read the data in and give appropriate names
for the columns.

\begin{Shaded}
\begin{Highlighting}[]
\KeywordTok{library}\NormalTok{(dplyr)}
\end{Highlighting}
\end{Shaded}

\begin{verbatim}
## 
## Attaching package: 'dplyr'
\end{verbatim}

\begin{verbatim}
## The following objects are masked from 'package:stats':
## 
##     filter, lag
\end{verbatim}

\begin{verbatim}
## The following objects are masked from 'package:base':
## 
##     intersect, setdiff, setequal, union
\end{verbatim}

\begin{Shaded}
\begin{Highlighting}[]
\NormalTok{insectsData <-}\StringTok{ }\KeywordTok{read.table}\NormalTok{(}\StringTok{"insects.txt"}\NormalTok{, }\DataTypeTok{header=}\OtherTok{FALSE}\NormalTok{, }\DataTypeTok{sep=}\StringTok{""}\NormalTok{)}
\end{Highlighting}
\end{Shaded}

\begin{Shaded}
\begin{Highlighting}[]
\NormalTok{insectsData <-}\StringTok{ }\NormalTok{insectsData }\OperatorTok\StringTok{ }\KeywordTok{rename}\NormalTok{(}\DataTypeTok{logdose =}\NormalTok{ V1, }\DataTypeTok{size =}\NormalTok{ V2, }\DataTypeTok{died =}\NormalTok{ V3)}

\CommentTok{#names(insectsData)}
\end{Highlighting}
\end{Shaded}

\begin{Shaded}
\begin{Highlighting}[]
\KeywordTok{head}\NormalTok{(insectsData)}
\end{Highlighting}
\end{Shaded}

\begin{verbatim}
##   logdose size died
## 1       1  250   28
## 2       2  250   53
## 3       3  250   93
## 4       4  250  126
## 5       5  250  172
## 6       6  250  197
\end{verbatim}

\begin{Shaded}
\begin{Highlighting}[]
\KeywordTok{attach}\NormalTok{(insectsData)}

\CommentTok{#knowns}
\NormalTok{n <-}\StringTok{ }\DecValTok{1500}
\NormalTok{groupn <-}\StringTok{ }\DecValTok{250}
\NormalTok{groups  <-}\StringTok{ }\DecValTok{6}
\end{Highlighting}
\end{Shaded}

\hypertarget{q1-a-plot-the-sample-log-odds-against-the-log-dose-x.}{%
\subsection{Q1 (a) Plot the sample log odds against the log dose,
x.}\label{q1-a-plot-the-sample-log-odds-against-the-log-dose-x.}}

Does this plot suggest fitting a logistic regression model to be
appropriate?

\begin{Shaded}
\begin{Highlighting}[]
\NormalTok{prop<-died}\OperatorTok{/}\NormalTok{size}

\KeywordTok{plot}\NormalTok{(logdose, }\KeywordTok{log}\NormalTok{(prop}\OperatorTok{/}\NormalTok{(}\DecValTok{1}\OperatorTok{-}\NormalTok{prop)), }\DataTypeTok{xlab=} \StringTok{"Log Dose"}\NormalTok{, }\DataTypeTok{ylab=}\StringTok{"sample log odds"}\NormalTok{)}
\end{Highlighting}
\end{Shaded}

\includegraphics{Module09HW_files/figure-latex/q1a-1-1.pdf}

\textbf{Answer Q1a}: Yes, this plot looks perfect for a logistic
regression model because the relationship is almost perfectly linear.

\hypertarget{q1-b-use-r-to-fit-the-logistic-regression-model-and-write-the-estimated-logistic-regression-equation.}{%
\subsection{Q1 (b) Use R to fit the logistic regression model, and write
the estimated logistic regression
equation.}\label{q1-b-use-r-to-fit-the-logistic-regression-model-and-write-the-estimated-logistic-regression-equation.}}

\begin{Shaded}
\begin{Highlighting}[]
\NormalTok{modelInsects <-}\StringTok{ }\KeywordTok{glm}\NormalTok{(prop}\OperatorTok{~}\NormalTok{logdose, }\DataTypeTok{family=}\StringTok{"binomial"}\NormalTok{, }\DataTypeTok{weights=}\NormalTok{size) }\CommentTok{# fit model. The weight is the sample/group size}
\KeywordTok{summary}\NormalTok{(modelInsects)}
\end{Highlighting}
\end{Shaded}

\begin{verbatim}
## 
## Call:
## glm(formula = prop ~ logdose, family = "binomial", weights = size)
## 
## Deviance Residuals: 
##       1        2        3        4        5        6  
## -0.5092  -0.1115   0.7461  -0.2869   0.4744  -0.5599  
## 
## Coefficients:
##             Estimate Std. Error z value Pr(>|z|)    
## (Intercept) -2.64367    0.15610  -16.93   <2e-16 ***
## logdose      0.67399    0.03911   17.23   <2e-16 ***
## ---
## Signif. codes:  0 '***' 0.001 '**' 0.01 '*' 0.05 '.' 0.1 ' ' 1
## 
## (Dispersion parameter for binomial family taken to be 1)
## 
##     Null deviance: 383.0695  on 5  degrees of freedom
## Residual deviance:   1.4491  on 4  degrees of freedom
## AIC: 39.358
## 
## Number of Fisher Scoring iterations: 3
\end{verbatim}

\textbf{Answer Q1b}

E(y) = exp(-2.64 + (0.67 * logdose)) / ( 1 + exp(-2.64 + (0.67 *
logdose)))

or

log(pi / (1 - pi)) = -2.64 + (0.67 * logdose)

\hypertarget{q1-c-interpret-the-estimated-coefficient-ux3b2ux2c61-in-context.}{%
\subsection{Q1 (c) Interpret the estimated coefficient βˆ1 in
context.}\label{q1-c-interpret-the-estimated-coefficient-ux3b2ux2c61-in-context.}}

\begin{Shaded}
\begin{Highlighting}[]
\KeywordTok{exp}\NormalTok{(}\FloatTok{0.67}\NormalTok{)}
\end{Highlighting}
\end{Shaded}

\begin{verbatim}
## [1] 1.954237
\end{verbatim}

The predicted odds of an insect dying is multiplied by exp(0.67) == 1.95
for a 1 unit increase in log dose.

or

The predicted odds ratio of an insect dying for a one unit increase in
log dose is 1.95.

\hypertarget{q1-d-what-are-the-estimated-odds-of-death-at-log-dose-level-x-2}{%
\subsection{Q1 (d) What are the estimated odds of death at log dose
level x =
2?}\label{q1-d-what-are-the-estimated-odds-of-death-at-log-dose-level-x-2}}

\begin{Shaded}
\begin{Highlighting}[]
\NormalTok{x_logdose <-}\StringTok{ }\DecValTok{2}

\NormalTok{oddsOfDeath <-}\StringTok{ }\KeywordTok{exp}\NormalTok{(}\OperatorTok{-}\FloatTok{2.64} \OperatorTok{+}\StringTok{ }\NormalTok{(}\FloatTok{0.67} \OperatorTok{*}\StringTok{ }\NormalTok{x_logdose))}
\KeywordTok{print}\NormalTok{(}\KeywordTok{paste}\NormalTok{(}\StringTok{"Odds of Death: "}\NormalTok{,oddsOfDeath))}
\end{Highlighting}
\end{Shaded}

\begin{verbatim}
## [1] "Odds of Death:  0.272531793034013"
\end{verbatim}

\hypertarget{q1-e-predict-the-probability-of-death-from-exposure-at-log-dose-level-x-2.}{%
\subsection{Q1 (e) Predict the probability of death from exposure at log
dose level x =
2.}\label{q1-e-predict-the-probability-of-death-from-exposure-at-log-dose-level-x-2.}}

\begin{Shaded}
\begin{Highlighting}[]
\CommentTok{#probOfDeathStart <- exp(-2.64 + (0.67 * x_logdose)) / ( 1 + exp(-2.64 + (0.67 * x_logdose)))}

\NormalTok{probDeath <-}\StringTok{ }\NormalTok{oddsOfDeath }\OperatorTok{/}\StringTok{ }\NormalTok{(}\DecValTok{1} \OperatorTok{+}\StringTok{ }\NormalTok{oddsOfDeath)}
\KeywordTok{print}\NormalTok{(}\KeywordTok{paste}\NormalTok{(}\StringTok{"Probability of Death: "}\NormalTok{,probDeath))}
\end{Highlighting}
\end{Shaded}

\begin{verbatim}
## [1] "Probability of Death:  0.214165016957441"
\end{verbatim}

\hypertarget{q1-f-carry-out-both-the-pearsons-ux3c72-and-deviance-goodness-of-fit-tests-to-check-the-fit}{%
\subsection{Q1 (f) Carry out both the Pearson's χ2 and deviance goodness
of fit tests to check the fit
\ldots{}}\label{q1-f-carry-out-both-the-pearsons-ux3c72-and-deviance-goodness-of-fit-tests-to-check-the-fit}}

of this logistic regression model. Clearly state the null and
alternative hypotheses, test statistic, and conclusion.

H0: The model fits well Ha: The model does not fit well

First, Pearson's

\begin{Shaded}
\begin{Highlighting}[]
\CommentTok{#Pearsons}
\NormalTok{pearson<-}\KeywordTok{residuals}\NormalTok{(modelInsects,}\DataTypeTok{type=}\StringTok{"pearson"}\NormalTok{)}
\NormalTok{X2<-}\KeywordTok{sum}\NormalTok{(pearson}\OperatorTok{^}\DecValTok{2}\NormalTok{)}
\NormalTok{X2 }\CommentTok{# test statistic is 1.45}
\end{Highlighting}
\end{Shaded}

\begin{verbatim}
## [1] 1.451786
\end{verbatim}

\begin{Shaded}
\begin{Highlighting}[]
\KeywordTok{print}\NormalTok{(}\KeywordTok{paste}\NormalTok{(}\StringTok{"Pearson test statistic"}\NormalTok{,X2))}
\end{Highlighting}
\end{Shaded}

\begin{verbatim}
## [1] "Pearson test statistic 1.4517864549869"
\end{verbatim}

\begin{Shaded}
\begin{Highlighting}[]
\NormalTok{n=}\StringTok{ }\DecValTok{6}
\NormalTok{p=}\DecValTok{2}

\NormalTok{pValue <-}\StringTok{ }\DecValTok{1}\OperatorTok{-}\KeywordTok{pchisq}\NormalTok{(X2,n}\OperatorTok{-}\NormalTok{p) }\CommentTok{# df = n-p, this gives p value}
\KeywordTok{print}\NormalTok{(}\KeywordTok{paste}\NormalTok{(}\StringTok{"pvalue: "}\NormalTok{,pValue))}
\end{Highlighting}
\end{Shaded}

\begin{verbatim}
## [1] "pvalue:  0.835146184673398"
\end{verbatim}

Second, Deviance GOF

\begin{Shaded}
\begin{Highlighting}[]
\DecValTok{1} \OperatorTok{-}\StringTok{ }\KeywordTok{pchisq}\NormalTok{(modelInsects}\OperatorTok{$}\NormalTok{deviance, n}\OperatorTok{-}\NormalTok{p) }
\end{Highlighting}
\end{Shaded}

\begin{verbatim}
## [1] 0.8356191
\end{verbatim}

\textbf{Answer q1f}: We fail reject the null hypothesis, the model is a
good fit because p value \textgreater{} .05 for both GOF tests. The
model fits well which agrees with our analysis of the plot of sample log
odds against log dose.

\hypertarget{question-2.-no-r-needed-a-health-clinic-sent}{%
\section{Question 2. (No R needed) A health clinic sent
\ldots{}}\label{question-2.-no-r-needed-a-health-clinic-sent}}

fliers to its clients to encourage everyone to get a flu shot. In a
follow-up study, 159 elderly clients were randomly selected and asked if
they received a flu shot.

A client who received a flu shot was coded y = 1, and a client who did
not receive a flu shot was coded y = 0.

Data were also collected on their

\begin{itemize}
\tightlist
\item
  age, x1,
\item
  health awareness rating on a 0-100 scale (higher values indicate
  greater awareness), x2,
\item
  and gender, x3, where males were coded x3 = 1 and females were coded
  x3 = 0.
\end{itemize}

A first order logistic regression model was fitted and the output is
displayed below. \ldots{}

\hypertarget{q2-a-interpret-the-estimated-coefficient-for-x3-gender-in-context.}{%
\subsection{Q2 (a) Interpret the estimated coefficient for x3, gender,
in
context.}\label{q2-a-interpret-the-estimated-coefficient-for-x3-gender-in-context.}}

\textbf{Work on q2a}:

B3 = 0.43397.

\begin{Shaded}
\begin{Highlighting}[]
\KeywordTok{exp}\NormalTok{(}\FloatTok{0.43397}\NormalTok{)}
\end{Highlighting}
\end{Shaded}

\begin{verbatim}
## [1] 1.543373
\end{verbatim}

\textbf{Answer q2a}: When age, and awareness rating are held constant,
the odds of an elderly male patient receiving the flu shot is
exp(0.43397), or 1.543373 times the odds of a an elderly female patient
receiving the flu shot.

\hypertarget{q2-b-conduct-the-wald-test-for-b3.}{%
\subsection{Q2 (b) Conduct the Wald test for B3.
\ldots{}}\label{q2-b-conduct-the-wald-test-for-b3.}}

State the null and alternative hypotheses, calculate the test statistic,
and make a conclusion in context.

\textbf{Work on q2b}

H0: The coef for gender, B3, equals 0 Ha: The coef for gender, B3, does
not equal 0

Wald test statistic: Bj-hat - 0 / se(Bj-hat)

\begin{Shaded}
\begin{Highlighting}[]
\NormalTok{Bjhat <-}\StringTok{ }\FloatTok{0.43397}
\NormalTok{seBjhat <-}\StringTok{ }\FloatTok{0.52179}

\NormalTok{wTestStatistic <-}\StringTok{ }\NormalTok{Bjhat }\OperatorTok{/}\StringTok{ }\NormalTok{seBjhat}
\KeywordTok{print}\NormalTok{(}\KeywordTok{paste}\NormalTok{(}\StringTok{"Wald test statistic for B3: "}\NormalTok{, wTestStatistic))}
\end{Highlighting}
\end{Shaded}

\begin{verbatim}
## [1] "Wald test statistic for B3:  0.831694743095882"
\end{verbatim}

\begin{Shaded}
\begin{Highlighting}[]
\CommentTok{#compare to pnorm}
\NormalTok{wPValue <-}\StringTok{ }\NormalTok{(}\DecValTok{1}\OperatorTok{-}\KeywordTok{pnorm}\NormalTok{(wTestStatistic))}\OperatorTok{*}\DecValTok{2}
\KeywordTok{print}\NormalTok{(}\KeywordTok{paste}\NormalTok{(}\StringTok{"Wald p value for B3: "}\NormalTok{, wPValue))}
\end{Highlighting}
\end{Shaded}

\begin{verbatim}
## [1] "Wald p value for B3:  0.405581268667017"
\end{verbatim}

\textbf{Answer q2b}: We do not reject the null hypothesis, gender is not
statistically significant in our model.

\hypertarget{q2-c-calculate-a-95-confidence-interval-for-b3-and-interpret-the-interval-in-context.}{%
\subsection{Q2 (c) Calculate a 95\% confidence interval for B3, and
interpret the interval in
context.}\label{q2-c-calculate-a-95-confidence-interval-for-b3-and-interpret-the-interval-in-context.}}

\textbf{Work on q2c}

\begin{Shaded}
\begin{Highlighting}[]
\NormalTok{zValue <-}\StringTok{ }\FloatTok{1.96} \CommentTok{# for 95% CI}

\NormalTok{CILow <-}\StringTok{ }\NormalTok{Bjhat }\OperatorTok{-}\StringTok{ }\NormalTok{(zValue }\OperatorTok{*}\StringTok{ }\NormalTok{seBjhat)}

\NormalTok{CIHigh <-}\StringTok{ }\NormalTok{Bjhat }\OperatorTok{+}\StringTok{ }\NormalTok{(zValue }\OperatorTok{*}\StringTok{ }\NormalTok{seBjhat)}

\KeywordTok{print}\NormalTok{(}\KeywordTok{paste}\NormalTok{(}\StringTok{"The 95% CI for B3 is: "}\NormalTok{, CILow, }\StringTok{" - "}\NormalTok{, CIHigh))}
\end{Highlighting}
\end{Shaded}

\begin{verbatim}
## [1] "The 95% CI for B3 is:  -0.5887384  -  1.4566784"
\end{verbatim}

\begin{Shaded}
\begin{Highlighting}[]
\KeywordTok{print}\NormalTok{(}\KeywordTok{paste}\NormalTok{(}\StringTok{"Low odds ratio: "}\NormalTok{, }\KeywordTok{exp}\NormalTok{(CILow)))}
\end{Highlighting}
\end{Shaded}

\begin{verbatim}
## [1] "Low odds ratio:  0.555027065365734"
\end{verbatim}

\begin{Shaded}
\begin{Highlighting}[]
\KeywordTok{print}\NormalTok{(}\KeywordTok{paste}\NormalTok{(}\StringTok{"High odds ratio: "}\NormalTok{, }\KeywordTok{exp}\NormalTok{(CIHigh)))}
\end{Highlighting}
\end{Shaded}

\begin{verbatim}
## [1] "High odds ratio:  4.2916805807803"
\end{verbatim}

When age, and awareness rating are held constant, the odds of an elderly
male patient receiving the flu shot is between -0.589 times and 1.46
time the odds of a an elderly female receiving the flu shot. 0 is within
the CI, which supports the conclusion that gender is not a significant
predictor.

\hypertarget{q2-d-comment-on-whether-your-conclusions-from-parts-2b-and-2c-are-consistent.}{%
\subsection{Q2 (d) Comment on whether your conclusions from parts 2b and
2c are
consistent.}\label{q2-d-comment-on-whether-your-conclusions-from-parts-2b-and-2c-are-consistent.}}

\textbf{Answer q2d}: 0 is within the CI, which supports the conclusion
that gender is not a significant predictor.

\hypertarget{q2-e-suppose-you-want-to-drop-the-coefficients-for-age-and-gender-b1-and-b3.}{%
\subsection{Q2 (e) Suppose you want to drop the coefficients for age and
gender, B1 and
B3.}\label{q2-e-suppose-you-want-to-drop-the-coefficients-for-age-and-gender-b1-and-b3.}}

A logistic regression model for just awareness was fitted, and the
output is shown below. \ldots{}

Carry out the appropriate hypothesis test to see if the coefficients for
age and gender can be dropped.

\textbf{Work on q2e}

This is analogous to a partial F test in linear regression, but
comparing deviance of the partial and full model.

H0: The coefs B1 and B3 are equal to 0. Ha: At least one of the coef B1
or B3 is not equal to 0.

\begin{Shaded}
\begin{Highlighting}[]
\NormalTok{partialDeviance <-}\StringTok{ }\FloatTok{113.2}

\NormalTok{fullDeviance <-}\StringTok{ }\FloatTok{105.09}

\NormalTok{df <-}\StringTok{ }\DecValTok{2} \CommentTok{# considering removing 2 predictors}

\NormalTok{deltaG2 <-}\StringTok{ }\NormalTok{partialDeviance }\OperatorTok{-}\StringTok{ }\NormalTok{fullDeviance}

\DecValTok{1}\OperatorTok{-}\KeywordTok{pchisq}\NormalTok{(deltaG2, df)}
\end{Highlighting}
\end{Shaded}

\begin{verbatim}
## [1] 0.01733548
\end{verbatim}

\begin{Shaded}
\begin{Highlighting}[]
\KeywordTok{qchisq}\NormalTok{(.}\DecValTok{025}\NormalTok{, df) }\CommentTok{# critical value}
\end{Highlighting}
\end{Shaded}

\begin{verbatim}
## [1] 0.05063562
\end{verbatim}

\textbf{Conclusion q2e}: We reject the null hypothesis because p-value,
0.017, is \textless{} .05. We must use the full model when comparing to
the awareness-only model.

\hypertarget{q2-f-based-on-your-conclusion-in-question-2e-what-are-the-estimated-odds-of-a-client-receiving-the}{%
\subsection{Q2 (f) Based on your conclusion in question 2e, what are the
estimated odds of a client receiving the
\ldots{}}\label{q2-f-based-on-your-conclusion-in-question-2e-what-are-the-estimated-odds-of-a-client-receiving-the}}

flu shot if the client is 70 years old, has a health awareness rating of
65, and is male?

What is the estimated probability of this client receiving the flu shot?

\textbf{Work on q2f}

Use the reduced model with only B2

\begin{Shaded}
\begin{Highlighting}[]
\NormalTok{x_age <-}\StringTok{ }\DecValTok{70}
\NormalTok{x_gender <-}\StringTok{ }\DecValTok{1}
\NormalTok{x_awareness <-}\StringTok{ }\DecValTok{65}

\NormalTok{linearPart <-}\StringTok{ }\FloatTok{-1.17716} \OperatorTok{+}\StringTok{ }\NormalTok{(}\FloatTok{0.07279} \OperatorTok{*}\StringTok{ }\NormalTok{x_age) }\OperatorTok{-}\StringTok{ }\NormalTok{(}\FloatTok{0.09899} \OperatorTok{*}\StringTok{ }\NormalTok{x_awareness) }\OperatorTok{+}\StringTok{ }\NormalTok{(}\FloatTok{0.43397} \OperatorTok{*}\StringTok{ }\NormalTok{x_gender)}
\KeywordTok{print}\NormalTok{(}\KeywordTok{paste}\NormalTok{(}\StringTok{"log odds: "}\NormalTok{, linearPart))}
\end{Highlighting}
\end{Shaded}

\begin{verbatim}
## [1] "log odds:  -2.08224"
\end{verbatim}

\begin{Shaded}
\begin{Highlighting}[]
\KeywordTok{print}\NormalTok{(}\KeywordTok{paste}\NormalTok{(}\StringTok{"odds: "}\NormalTok{, }\KeywordTok{exp}\NormalTok{(linearPart)))}
\end{Highlighting}
\end{Shaded}

\begin{verbatim}
## [1] "odds:  0.124650681714281"
\end{verbatim}

\begin{Shaded}
\begin{Highlighting}[]
\NormalTok{Ey1 <-}\StringTok{ }\KeywordTok{exp}\NormalTok{(linearPart) }\OperatorTok{/}\StringTok{ }\NormalTok{(}\DecValTok{1} \OperatorTok{+}\StringTok{ }\KeywordTok{exp}\NormalTok{(linearPart))}
\NormalTok{Ey1 }\CommentTok{#prob}
\end{Highlighting}
\end{Shaded}

\begin{verbatim}
## [1] 0.110835
\end{verbatim}

\begin{Shaded}
\begin{Highlighting}[]
\KeywordTok{print}\NormalTok{(}\KeywordTok{paste}\NormalTok{(}\StringTok{"prob: "}\NormalTok{, Ey1))}
\end{Highlighting}
\end{Shaded}

\begin{verbatim}
## [1] "prob:  0.110835020812221"
\end{verbatim}

\textbf{Answer q2f}: I estimate the odds as .125 that this patient
received a flu shot.

\hypertarget{q2-g-based-on-your-conclusion-in-question-2e}{%
\subsection{Q2 (g) Based on your conclusion in question 2e,
\ldots{}}\label{q2-g-based-on-your-conclusion-in-question-2e}}

interpret the estimated coefficient for age, in context.

\begin{Shaded}
\begin{Highlighting}[]
\KeywordTok{exp}\NormalTok{(}\FloatTok{0.07279}\NormalTok{)}
\end{Highlighting}
\end{Shaded}

\begin{verbatim}
## [1] 1.075505
\end{verbatim}

\textbf{Interpretation of the coefficient for age}: When awareness and
gender are held constant, each increase in 1 year of age multiplies the
predicted odds of receiving a flu shot by 1.076

What does this interpretation tell us about the estimated probability of
a client receiving the flu shot if the client is 71 years old, has a
health awareness rating of 65, and is male, compared to the estimated
probability found for a similar client who is a year younger?

\textbf{Answer q2g}:

I would expect the odds of the similar patient, except 1 year older than
70 to be 12.5 + (12.5 * .076)\% \textasciitilde= 13.4\%

And to compare probabilities we exam the odds ratio == exp(B-hat) ==
1.075505 and is the increase in probability of flu shot, when gender and
awareness are constant, of a 1 year increase in age.

\textbf{And to confirm}

\begin{Shaded}
\begin{Highlighting}[]
\NormalTok{x_age <-}\StringTok{ }\DecValTok{71}
\NormalTok{x_gender <-}\StringTok{ }\DecValTok{1}
\NormalTok{x_awareness <-}\StringTok{ }\DecValTok{65}

\NormalTok{linearPart71 <-}\StringTok{ }\FloatTok{-1.17716} \OperatorTok{+}\StringTok{ }\NormalTok{(}\FloatTok{0.07279} \OperatorTok{*}\StringTok{ }\NormalTok{x_age) }\OperatorTok{-}\StringTok{ }\NormalTok{(}\FloatTok{0.09899} \OperatorTok{*}\StringTok{ }\NormalTok{x_awareness) }\OperatorTok{+}\StringTok{ }\NormalTok{(}\FloatTok{0.43397} \OperatorTok{*}\StringTok{ }\NormalTok{x_gender)}
\KeywordTok{exp}\NormalTok{(linearPart71)}
\end{Highlighting}
\end{Shaded}

\begin{verbatim}
## [1] 0.1340624
\end{verbatim}

\end{document}
